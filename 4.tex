\chapter{多项式的根}
多项式的根是一个重要的概念,也是我们研究的主要问题之一,本章在复习的基础上,将系统地研究多项式的整数根,有理数根,实数根的存在、判定和计算方法。

\section{多项式的根及求根公式}
我们已经知道,如果当$x=a$时,多项式的值$f(a)=0$, 就把数值$a$叫做多项式$f(x)$的一个根,或叫做多项式函数$f(x)$的一个零点。

显然,零次多项式$f(x)=b\ne 0$, 没有任何根;零多项式$f(x)=0$有无限多个根(任意数都是它的根)。

因此,一元$n$次多项式
\[f(x)=a_nx^n+a_{n-1}x^{n-1}+\cdots+a_1x+a_0\quad (a_n\ne 0)\]
的根,就是一元$n$次方程
\[a_nx^n+a_{n-1}x^{n-1}+\cdots+a_1x+a_0=0\]
的根。求多项式的根,就是解相应的方程式。

要求多项式的根,首先应明确在那一个数系范围内,因为多项式和方程一样,同一个多项式在不同的数系范围内,可能会有不同的根存在。

在此,我们主要讨论和计算多项式的实数根,对其中更容易研究的有理系数多项式的有理根及整系数多项式的整数根更要特别讨论。

\subsection{一元一、二次多项式的求根公式}

一元一次、一元
二次多项式的求根公式也就是一元一次、一元二次方程的求根的公式,我们早已在初中代数中学习过,现就一般形式总结如下:

\begin{enumerate}
    \item 一元一次多项式$  f (x) =ax+b\quad  (a\ne 0)$
    有且只有一个实数根:
    \begin{equation}
        x=-\frac{b}{a}
    \end{equation}
    \item 一元二次多项式
    $f (x) =ax^2+bx+c\quad  (a\ne0)$
    当且仅当$b^2-4ac\ge 0$时,有两个(不同或相同)实根
    \begin{equation}
        x=\frac{-b\pm\sqrt{b^2-4ac}}{2a}
    \end{equation}
    当且仅当$b^2-4ac<0$时,没有实数根。
\end{enumerate}

(4.1), (4.2)就是一次和二次多项式的求根公式,也叫
做多项式的公式解,它显示了用多项式的各项系数通过加、减、乘(乘方)、除、开方运算,就可以求得它的根。

由求根公式(4.1)可以知道,对于一次多项式来说,它的根仅仅是系数进行减、除运算,因而,由有理数系对加、减、乘、除运算的封闭性就得出:

有理系数一次多项式,一定有一个有理数根,而由整数系对除法运算的不封闭性就得出:
整系数一次多项式,不一定有整数根。

例如,$f(x)=3x-2$,有一个有理根$x=\frac{2}{3}$,
但没有整数根。

还应指出,对于系数为参数的多项式$\varphi_1(x)=ax+b$
的根,可进行一般性的全面讨论:
\begin{enumerate}
    \item 当$a\ne 0$时,不论$b$为任何数,$\varphi_1(x)$都有唯一实
    数根:$x=-\frac{b}{a}$;
    \item 当$a=0$, 但$b\ne 0$时,$\varphi_1(x)$为零次多项式,它没有任何根;
    \item 当$a=b=0$时,$\varphi_1(x)=0$为零多项式,它有无限多个根。
\end{enumerate}

\begin{example}
    试讨论$g(x)=2(a+b)x-(a+b)^2$的根的情形,如有根存在,求出根。
\end{example}

\begin{solution}
    由多项式根的定义知,
$2 (a+b) a- (a+b)^2=0$
即
\[2 (a+b) x= (a+b)^2\]
\begin{itemize}
    \item 当$a+b\ne 0$时,$g(x)$有唯一实根:$x=\frac{a+b}{2}$;
    \item 当$a+b=0$, 即$a=-b$时,$g(x)=0$, 它有无限多个根,即任意实数都是它的根。
\end{itemize}
\end{solution}

由求根公式(4.2)同样可以知道,对于二次多项式来说,它的根要通过系数的加、减、乘(乘方)、除法运算以及开平方运算而求出,因而,由有理数系对开平方运算的不封闭性,就得出:

有理系数二次多项式,不一定有有理数根存在。更不一定有整数根存在。

事实上,即便是整系数二次多项式,也不一定有整数根或有理根,甚至没有实数根。

讨论一元二次多项式的根,和一元二次方程根的讨论一样,可由判别式$b^2-4ac$的符号分为三种情形:
\begin{enumerate}
    \item 当$b^2-4ac>0$时,$f(x)$有两个不同实根;
    \item 当$b^2-4ac=0$时,$f(x)$有两个相同实根;
    \item 当$b^2-4ac<0$时,$f(x)$没有实根。
\end{enumerate}

同样地对于系数为参数的多项式
$\varphi_2 (x) =ax^2+bx+c$
的根,也可以系统全面的讨论如下:













