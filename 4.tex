\chapter{多项式的根}
多项式的根是一个重要的概念,也是我们研究的主要问题之一,本章在复习的基础上,将系统地研究多项式的整数根,有理数根,实数根的存在、判定和计算方法。

\section{多项式的根及求根公式}
我们已经知道,如果当$x=a$时,多项式的值$f(a)=0$, 就把数值$a$叫做多项式$f(x)$的一个根,或叫做多项式函数$f(x)$的一个零点。

显然,零次多项式$f(x)=b\ne 0$, 没有任何根;零多项式$f(x)=0$有无限多个根(任意数都是它的根)。

因此,一元$n$次多项式
\[f(x)=a_nx^n+a_{n-1}x^{n-1}+\cdots+a_1x+a_0\quad (a_n\ne 0)\]
的根,就是一元$n$次方程
\[a_nx^n+a_{n-1}x^{n-1}+\cdots+a_1x+a_0=0\]
的根。求多项式的根,就是解相应的方程式。

要求多项式的根,首先应明确在那一个数系范围内,因为多项式和方程一样,同一个多项式在不同的数系范围内,可能会有不同的根存在。

在此,我们主要讨论和计算多项式的实数根,对其中更容易研究的有理系数多项式的有理根及整系数多项式的整数根更要特别讨论。

\subsection{一元一、二次多项式的求根公式}

一元一次、一元
二次多项式的求根公式也就是一元一次、一元二次方程的求根的公式,我们早已在初中代数中学习过,现就一般形式总结如下:

\begin{enumerate}
    \item 一元一次多项式$  f (x) =ax+b\quad  (a\ne 0)$
    有且只有一个实数根:
    \begin{equation}
        x=-\frac{b}{a}
    \end{equation}
    \item 一元二次多项式
    $f (x) =ax^2+bx+c\quad  (a\ne0)$
    当且仅当$b^2-4ac\ge 0$时,有两个(不同或相同)实根
    \begin{equation}
        x=\frac{-b\pm\sqrt{b^2-4ac}}{2a}
    \end{equation}
    当且仅当$b^2-4ac<0$时,没有实数根。
\end{enumerate}

(4.1), (4.2)就是一次和二次多项式的求根公式,也叫
做多项式的公式解,它显示了用多项式的各项系数通过加、减、乘(乘方)、除、开方运算,就可以求得它的根。

由求根公式(4.1)可以知道,对于一次多项式来说,它的根仅仅是系数进行减、除运算,因而,由有理数系对加、减、乘、除运算的封闭性就得出:

有理系数一次多项式,一定有一个有理数根,而由整数系对除法运算的不封闭性就得出:
整系数一次多项式,不一定有整数根。

例如,$f(x)=3x-2$,有一个有理根$x=\frac{2}{3}$,
但没有整数根。

还应指出,对于系数为参数的多项式$\varphi_1(x)=ax+b$
的根,可进行一般性的全面讨论:
\begin{enumerate}
    \item 当$a\ne 0$时,不论$b$为任何数,$\varphi_1(x)$都有唯一实
    数根:$x=-\frac{b}{a}$;
    \item 当$a=0$, 但$b\ne 0$时,$\varphi_1(x)$为零次多项式,它没有任何根;
    \item 当$a=b=0$时,$\varphi_1(x)=0$为零多项式,它有无限多个根。
\end{enumerate}

\begin{example}
    试讨论$g(x)=2(a+b)x-(a+b)^2$的根的情形,如有根存在,求出根。
\end{example}

\begin{solution}
    由多项式根的定义知,
$2 (a+b) a- (a+b)^2=0$
即
\[2 (a+b) x= (a+b)^2\]
\begin{itemize}
    \item 当$a+b\ne 0$时,$g(x)$有唯一实根:$x=\frac{a+b}{2}$;
    \item 当$a+b=0$, 即$a=-b$时,$g(x)=0$, 它有无限多个根,即任意实数都是它的根。
\end{itemize}
\end{solution}

由求根公式(4.2)同样可以知道,对于二次多项式来说,它的根要通过系数的加、减、乘(乘方)、除法运算以及开平方运算而求出,因而,由有理数系对开平方运算的不封闭性,就得出:

有理系数二次多项式,不一定有有理数根存在。更不一定有整数根存在。

事实上,即便是整系数二次多项式,也不一定有整数根或有理根,甚至没有实数根。

讨论一元二次多项式的根,和一元二次方程根的讨论一样,可由判别式$b^2-4ac$的符号分为三种情形:
\begin{enumerate}
    \item 当$b^2-4ac>0$时,$f(x)$有两个不同实根;
    \item 当$b^2-4ac=0$时,$f(x)$有两个相同实根;
    \item 当$b^2-4ac<0$时,$f(x)$没有实根。
\end{enumerate}

同样地对于系数为参数的多项式
$\varphi_2 (x) =ax^2+bx+c$
的根,也可以系统全面的讨论如下:















\begin{example}
    
\end{example}

\begin{solution}
    
\end{solution}

\begin{example}
    
\end{example}


\begin{solution}
    
\end{solution}

\begin{solution}
    
\end{solution}

\begin{solution}
    
\end{solution}





\begin{example}
    
\end{example}

\begin{solution}
    
\end{solution}

\begin{solution}
    
\end{solution}

\begin{solution}
    
\end{solution}

\begin{example}
    
\end{example}

\begin{example}
    
\end{example}

\begin{example}
    
\end{example}

\begin{example}
    
\end{example}

\begin{example}
    
\end{example}

\begin{example}
    
\end{example}

\begin{example}
    
\end{example}

\begin{example}
    
\end{example}
















\begin{ex}
解下列方程组
\begin{enumerate}
    \item $\begin{cases}
        \sqrt{x}+\sqrt{y}=a\\
        x\cdot y=b
    \end{cases}\quad (a>0,\; b>0)$
    \item $\begin{cases}
        x+y=12\\ \sqrt{x+2}+\sqrt{y-1}=5
    \end{cases}$
    \item $\begin{cases}
        x^2+y^2+2(x+y)=3(xy+1)\\
        2(x^2+y^2)-xy=6(x+y)-4
    \end{cases}$
\end{enumerate}
\end{ex}

\section*{习题4.5}
\begin{enumerate}
    \item 解下列方程组:
\begin{multicols}{2}
\begin{enumerate}
    \item $\begin{cases}
        xy+36=0\\x+y=5
    \end{cases}$
    \item $\begin{cases}
        x-y=7\\x^2+y^2=85
    \end{cases}$
    \item $\begin{cases}
        x^2-y^2-3x+2y=10\\x+y=7
    \end{cases}$
    \item $\begin{cases}
        (x-2)^2+(y+3)^2=9\\ 3x-2y=6
    \end{cases}$
    \item $\begin{cases}
        4x^2-9y^2=15\\ 2x-3y=5
    \end{cases}$
    \item $\begin{cases}
        \sqrt{x+1}+\sqrt{y-1}=5\\ x+y=13
    \end{cases}$
    \item $\begin{cases}
        \frac{4}{x^2}+\frac{25}{y^2}=25\\
        \frac{2}{x}+\frac{5}{y}=1
    \end{cases}$
    \item $\begin{cases}
        \frac{y}{x}+\frac{2x}{y}=3\\
        2x+3y=4
    \end{cases}$

\end{enumerate}
\end{multicols}

\item \begin{enumerate}
    \item $m$取什么值时,方程组$\begin{cases}
        y^2=4x\\y=2x+m
    \end{cases}$
有两个相等的实数解?并求出这个解。

\item 在什么情况下,关于$x,y$的方程组
$\begin{cases}
    x+y=a\\xy=b
\end{cases}$
有实数解?没有实数解?
\end{enumerate}


\item 解方程组:
\begin{enumerate}
    \begin{multicols}{2}
    \item $\begin{cases}
        x^2+y^2=5\\y^2=4x
    \end{cases}$
    \item $\begin{cases}
        x^2+y^2=101\\xy=-10
    \end{cases}$
    \item $\begin{cases}
        x^2-y^2+(x-y)-6=0\\
        x^2+2xy+y^2-25=0
    \end{cases}$
    \item $\begin{cases}
        x^2+xy+y^2=7\\
        6x^2-5xy+y^2=0
    \end{cases}$
\end{multicols}
    \item $\begin{cases}
        x^2-5xy+6y^2=0\\
        (x-4)(y-1)+(x-3)(y-2)=0
    \end{cases}$
    \item $\begin{cases}
        2x^2+27xy+6y^2-6x-21y-14=0\\
        2x^2-9xy-3y^2-6x+6y+4=0
    \end{cases}$
\end{enumerate}


\item 解方程组:
\begin{multicols}{2}
\begin{enumerate}
    \item $\begin{cases}
        (x-2)^2+(y-1)^2=25\\
        2(x-2)^2-3(y-1)^2=5
    \end{cases}$
    \item $\begin{cases}
        (x+3)^2+y^2=9\\
        9(x-2)^2+4y^2=36
    \end{cases}$
    \item $\begin{cases}
        x^2+y^2=8\\
        (x+1)^2=(y-1)^2
    \end{cases}$
    \item $\begin{cases}
        x^2-xy-2y^2+y=0\\
        x^2-xy-2y^2-3x+6y=0\\
    \end{cases}$
    \item $\begin{cases}
        x^2+y^2=x+y+20\\
        xy+10=2(x+y)
    \end{cases}$
    \item $\begin{cases}
        (x+y+1)^2+(x+y)^2=25\\
        x^2-y^2=8
    \end{cases}$
    \item $\begin{cases}
        2x^2-4xy+3y^2=36\\
        3x^2-4xy+2y^2=36
    \end{cases}$
    \item $\begin{cases}
        4x^2+9y^2=10\\
        2xy=1
    \end{cases}$
\end{enumerate}    
\end{multicols}

\item 解下列方程组:
\begin{enumerate}
    \item $\begin{cases}
      2x^2-xy-y^2+3x+2y=3\\
      x^2-3x+2=0  
    \end{cases}$
    \item $\begin{cases}
        x^2-15xy-3y^2+2x+9y-98=0\\
        5xy+y^2-3y+21=0
    \end{cases}$
    \item $\begin{cases}
        x^2+y^2+3xy-4x-4y+3=0\\
        xy+2x+2y-5=0
    \end{cases}$
    \item $\begin{cases}
        xy=3\\yz=6\\xz=2
    \end{cases}$
    \item $\begin{cases}
        \sqrt{\frac{x}{y}}+\sqrt{\frac{y}{z}}=\frac{5}{2}\\
        x+y=10
    \end{cases}$
    \item $\begin{cases}
        5x^2-6xy+5y^2=29\\
        7x^2-8xy+7y^2=43
    \end{cases}$
\end{enumerate}


\end{enumerate}

\section*{本章内容要点}


一、本章主要内容是讨论实系数多项式的根,若$\alpha$满足$f(\alpha)=0$, 则$\alpha$叫多项式$f(x)$的根,多项式$f(x)$的根就是方程$f(x)=0$的根。

二、多项式$f(x)$的求根,就是解方程$f(x)=0$. 对于一元多项式$f(x)$: 一次、二次、三次、四次多项式都有求根公式(也称为根式解),而五次以上的一元多项式没有求根公式(不存在根式解)。

三、有理系数多项式$f(x)$若有有理根$\frac{p}{q}$, $(p,q)=1$则
必定有$p$能整除常数项$a_0$, $q$能整除首项系数$a_n$. 特别地,若$f(x)$有整数根$\alpha$, 则$\alpha|a_0$. 

因此,求有理系数多项式$f(x)$的有理根时,就可以首先找出$a_n$与$a_0$的因数,配成以$a_n$的因数为分母,以$a_0$的因数为分子的各种应有形式的有理分数就是所求有理根的范围;其次再用余式定理与综合除法逐个试算,确定所求多项式的有理根,特别地,有理系数多项式的整数根,只要在$a_0$的所有因数中试算,即可确定。

四、同时满足$f(\alpha)=0$, $g(\alpha)=0$的数$\alpha$, 叫做多项式$f(x)$与$g(x)$的公根。

两多项式$f(x)$与$g(x)$有公根的充要条件是它们有一次公因式;求两多项式的公根,一般只要求它们的最高公因式的根就可以;也可以先求其中一个多项式的根,再逐个代入另一多项式去试算,凡满足的,就是公根,否则就不是公根。

五、如果$\alpha$满足
$f(x)=(x-\alpha)^m\cdot q(x)$, 且$q(\alpha)\ne 0$。那么,$\alpha$就叫做$f(x)$的$m$重根。

$\alpha$是$f(x)$的二重根的充要条件是$\alpha$为$f(x)$, $f'(x)$的公根。

若$(f(x),f'(x))$含有$(x-\alpha)^{m-1}$的因式,则$\alpha$就是$f(x)$的$m$重根,也是$f'(x)$的$m-1$重根。

对于一个多项式$f(x)$, 如果它有重根,那么,$(f(x),f'(x))$就是非零次多项式,且不为零多项式。因而。
$\frac{f (x)}{(f (x) ,f' (x) )}=\varphi(x)$就是一个没有重根的多项式,而且$\varphi(x)$与$f(x)$有相同的根。

六 实系数多项式的实根,一般是用有理数近似值表示,求实根的近似值主要依据多项式函数的中间值定理,采用逼近的方法。一般地要顺序解决以下几个问题:
\begin{enumerate}
    \item 确定根界
    
    多项式$f(x)=x^n+a_{n-1}x^{n-1}+\cdots+a_1x+a_0$的所有根在$[-M,M]$之中,$M=1+|a_{n-1}|+\cdots+|a_1|+|a_0|$;
    
    \item 确定根的个数,根的定位
    
    计算史笃姆函数序列及变号数$W(-M)$, $W(M)$, 由克笃姆定理可确定:没有重根的多项式$f(x)$在$[-M,M]$中有
    $(W(-M)-W(M))$个实根,并可将每个根限定在一个确定的区间中。
    \item 计算每一实根的近似值
    
    运用秦九韶法,可以计算出在$(a,b)$中的实根的任意精确度的近似值。在具体计算过程中,主要使用了两种变换,方便和简化了运算。
 \end{enumerate}

    七、二元二次方程组分为两种类型,第(I)类型是基础,它的求解主要是采用代入消元法;第(II)类型,我们仅讨论了一些具有特点的特殊方程组的解法,其中主要是:
    \begin{enumerate}
        \item 可转化为第(I)类型求解的方程组,其转化的主要方法是因式分解;消去二次项,消去非二次项、再分解因式,总之是降次。
        \item 可转化为含有一元方程的方程组,其转化的主要方法是消去含有某一元的各项。实际就是消元。
        \item 可用换元法解的轮换对称方程组。
\end{enumerate}

至于一般的由两个二元二次方程组的方程组,如果不具有以上这些特点,其解法繁难,我们先不予讨论。以后可以使用几何法解决。


\section*{复习题四}
\begin{enumerate}
    \item 求下列多项式的有理根:
\begin{multicols}{2}
\begin{enumerate}
    \item $\poly{1,-1,-8,12}$
    \item $\poly{1,-11,18,-8}$
    \item $\poly{1,-7,-7,43,42}$
    \item $\poly{1,0,4,8,0,32}$
    \item $\poly{1,0,-1,2,-1}$
    \item $\poly{4,0,-9,6,-1}$
\end{enumerate}
\end{multicols}

\item 如果多项式$f(x)=ax^2+bx+c$的二根之比为$\frac{2}{3}$, 求证:$6b^2=25ac$。

\item 判别下列多项式有没有重根,若有,求出其重根。
\begin{enumerate}
    \item $\poly{1,0,-24,64,-48}$
    \item $\poly{1,-5,7,-2,4,-8}$
    \item $\poly{1,0,4,-4,-3}$
\end{enumerate}

\item 证明:$f(x)=1+\frac{x}{1!}+\frac{x^2}{2!}+\cdots+\frac{x^n}{n!}$没有重根。

\item 求$f(x)=x^4+x^3-2x-4$与$g(x)=x^4-x^3+2x-4$的公
根。

\item 求多项式$x^3+px+q$有重根的条件。

\item \begin{enumerate}
    \item 若$a$为整数,但$|a|\ne 2$, 试证:多项式
$f (x) =x^2+ax+1$没有有理根。
\item 若$a,b,c$都是奇数,试证明$f(x)=ax^2+bx+c$没有整数根。
\end{enumerate}


\item $a\ne 0$, 求证多项式$f(x)=x^n-a^n$没有重根。

\item 已知$f(x)=2x^4-8x^3+19x^2-12x+24$有两个根分别是
$g(x)=x^4-2x^3+3x^2-2x+2$的两个根的2倍,求这两个根。

\item 求多项式$p(x)=x^3+2x^2-5x-7$的正根的近似值,使
误差小于$10^{-8}$.

\item 求98的5次方根,使误差小于$10^{-6}$(允许应用四位对数
表求出前若干位数字)。

\item 求多项式$f(x)=x^3+x^2+x-10^{10}$的正根的近似值,使
误差小于1.

\item 将$x^5-243$写成以$x-3$为元的多项式,将$x^3+x^2+1$写
成以$x+1$为元的多项式。

\item 应用中间值定理,写出下列各多项式的实根在哪些连续
整数之间。
\begin{enumerate}
    \item $\poly{1,-2,-1,6,2}$
    \item $\poly{1,1,-2,1}$
    \item $\poly{1,-8,14,4,-6}$
\end{enumerate}

\item 若三次多项式$f(x)=x^3-2x^2-x+3$有三个根$\alpha,\beta,\gamma$,试求下列各式的值:
\begin{enumerate}
    \item $\alpha^2+\beta^2+\gamma^2$
    \item $\alpha^3+\beta^3+\gamma^3$
    \item $(\alpha+1)(\beta+1)(\gamma+1)$
    \item $\alpha^2(\beta+\gamma)+\beta^2(\alpha+\gamma)+\gamma^2(\alpha+\beta)$
\end{enumerate}


\item 应用多项式的第一种换元变形,使
$f (x) =x^3-6x^2+5x+7$变形为$g(y)$后,$g(y)$中$y$的系数为零。
\item 应用多项式的第一种换元变形,使
$f (x) =a_nx^n+a_{n-1}x^{n-1}+\cdots +a_1x+a_0\;  (a, \ne 0)$变形为$g(y)$后,$g(y)$中$y^{n-1}$的系数为零。

\item 解方程组:
\begin{multicols}{2}
  \begin{enumerate}
\item $\begin{cases}x^{2}-x y+y^{2}=48 \\ x-y-8=0\end{cases}$
\item $\begin{cases}x^{2}+3 x y+y^{2}=1 \\ 3 x^{2}+x y+3 y^{2}=13\end{cases}$
\item $\begin{cases}x^{2}+y^{2}+4 x-2 y+3=0 \\ x^{2}+4 x y-y^{2}+10 y-9=0\end{cases}$
\item $\begin{cases}x^{2}-x y=12 \\ x y-2 y^{2}=1\end{cases}$
\item $\begin{cases}\frac{1}{x^{2}}+\frac{1}{x y}=\frac{1}{a^{2}}\\
    \frac{1}{y^{2}}+\frac{1}{x y}=\frac{1}{b^{2}}
\end{cases}$
\item $\begin{cases}x^{4}+y^{4}=97 \\ x+y=5\end{cases}$
\item $2(x-y)+x y=3 x y-(x-y)=7$
\end{enumerate}  
\end{multicols}

\end{enumerate}